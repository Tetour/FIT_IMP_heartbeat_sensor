\documentclass[11pt,a4paper]{article}

% Packages
\usepackage[utf8x]{inputenc}
\usepackage[T1]{fontenc}
\usepackage[czech]{babel}
\usepackage{amsmath,amssymb,amsthm}
\usepackage{graphicx}
\usepackage{float}
\usepackage{hyperref}
\usepackage{listings}
\usepackage{xcolor}
\usepackage{geometry}
\usepackage{fancyhdr}
\usepackage{lastpage}

% Page geometry
\geometry{margin=1in}

% Disable paragraph indentation
\parindent=0pt

% Code highlighting
\lstset{
    language=C++,
    basicstyle=\ttfamily\footnotesize,
    keywordstyle=\color{blue},
    commentstyle=\color{green!60!black},
    stringstyle=\color{red},
    numbers=left,
    numberstyle=\tiny,
    frame=single,
    breaklines=true,
    captionpos=b
}

% Header and footer
\pagestyle{fancy}
\fancyhf{}
\fancyhead[L]{\leftmark}
\fancyhead[R]{\thepage\ z \pageref{LastPage}}

% Title page info
\title{Dokumentace - Kit snímače srdečního tepu}
\author{Adam Vrška (xvrskaa00)}
\date{\today}

\begin{document}

% Title page
\maketitle
\thispagestyle{empty}

\section{Úvod}

Tent projekt se zabývá implementací snímače srdečního tepu založeného na mikrokontroléru ESP32. Zařízení je schopno měřit srdeční tep v reálném čase pomocí pulzního senzoru, zobrazovat výsledky na OLED displeji a umožňuje uživatelskou interakci prostřednictvím joysticku. Data mohou být také zaznamenávána pro pozdější analýzu. 


\subsection{Hardwarové komponenty}

\begin{itemize}
    \item ESP32 WEMOS D1 R32 vývojová deska
    \item Senzor srdečního tepu
    \item OLED displej (SSD1306 128x64)
    \item Joystick modul pro navigaci
\end{itemize}

\subsubsection{Hardwarové zapojení}

Komponenty jsou zapojeny podle následujícího schématu:

\vspace{1em}

\begin{tabular}{|l|l|l|}
\hline
\textbf{Komponenta} & \textbf{Funkce} & \textbf{Pin ESP32} \\
\hline
Pulzní senzor & Analogový signál & GPIO 34 \\
\hline
OLED displej & I2C SDA & GPIO 21 \\
 & I2C SCL & GPIO 22 \\
\hline
Joystick & UP vstup & GPIO 18 \\
 & DOWN vstup & GPIO 19 \\
 & LEFT vstup & GPIO 23 \\
 & RIGHT vstup & GPIO 5 \\
 & PRESS vstup & GPIO 13 \\
\hline
\end{tabular}

\vspace{1em}

Všechny komponenty jsou napájeny z 3.3V výstupu ESP32.

\section{Softwarové řešení}

Zdrojový kód je rozdělen do jednotlivých tříd na základě hardwarových komponent, tedy tříd Sensor, Display a Joystick. Navíc existuje třída DataLogger pro správu záznamu dat. Součástí projektu jsou také podpůrné Python skripty pro ukládání a následné grafické zobrazení dat.

\subsection{Algoritmus detekce tepu}

Detekce tepu používá algoritmus adaptivního prahování, tedy dynamické úpravy prahové hodnoty na základě aktuálních vrcholů a údolí signálu. Hodnota prahu se používá na určení náběžné hrany, kterou je detekován srdeční pulz. Práh je určený jako stření hodnota mezi vrcholem a údolím signálu. Aby vrcholky a údolí nekonvergovali k extrémním hodnotá signálu, je implementován mechanismu skterý je z jejich extrémních hodnot samovolně snižuje, resp. zvyšuje. Tento přístup by měl zajistit spolehlivou detekci i při různých podmínkách měření.

\subsection{Uživatelské rozhraní}

\subsubsection{Zobrazení BPM}

V této sekci vidí uživatel pouze aktuální hodnotu srdečního tepu v BPM (údery za minutu). Kliknutím na střed joysticku se přepne na další obrazovku. Kliknutím do leva nebo doprava se spustí nebo zastaví záznam dat.

\subsubsection{Zobrazení grafu}

Tato obrazovka zobrazuje grafický průběh signálu ze senzoru srdečního tepu. Je zde také k dispozici aktuální hodnota siználu z tohoto senzoru. Stisknutím středu joysticku se přepne na obrazovku nastavení. Kliknutím do leva nebo doprava se spustí nebo zastaví záznam dat.

\subsubsection{Nastavení}

V sekci nastavení má uživatel přístup k různým konfiguračním možnostem:

\begin{itemize}
    \item \textbf{Recording}: Možnost zapnout nebo vypnout nahrávání dat.
    \item \textbf{Auto record}: Možnost zapnout automatické ukončení nahrávání po určitém čase a nastavení této doby.
    \item \textbf{Thrs offset}: Nastavení offsetu prahové hodnoty pro detekci tepu.
    \item \textbf{Decay rate}: Nastavení rycholosti snižování vrcholků a zvyšování údolí signálu v algoritmu adaptivního prahování.
\end{itemize}

Navigace v nastavení probíhá pomocí joysticku - nahoru/dolů pro výběr položky, doleva/doprava pro změnu hodnoty, střed pro přechod na obrazovku zobrazení BPM.

\subsection{Práce s aplikací}

Aplikace obsahuje Makefile pro snadné sestavení a práci s aplikací. Jednotlívé cíle významné pro uživatele jsou popsány v souboru README.md.

\section{Analýza dat}

\subsection{Záznam dat bez měření tepu}

Následující graf zobrazuje záznam dat bez přítomnosti srdečního tepu (šum/ruch):

\begin{figure}[H]
    \centering
    \includegraphics[width=\textwidth]{../../data/examples/noise_1.png}
    \caption{Záznam dat bez přítomného prstu pro měření tepu.}
    \label{fig:noise}
\end{figure}

\subsection{Záznam dat s měřením tepu}

Následující graf zobrazuje záznam dat s přítomným srdečním tepem:

\begin{figure}[H]
    \centering
    \includegraphics[width=\textwidth]{../../data/examples/hearthbeat_1.png}
    \caption{Záznam dat s přítomným prstem pro měření tepu.}
    \label{fig:heartbeat}
\end{figure}

\section{Závěr}

Data získaná v rámci projektu nedosahují konzistentní kvality, což může být způsobeno následujícími faktory:
\begin{itemize}
    \item Nesprávné umístění senzoru na prstu.
    \item Osvětlení okolí, které může ovlivnit optický senzor.
    \item Ovlivnění měření nahráváním dat v reálném čase.
    \item Nesprávným výstupem senzoru.
    \item Problémy s časováním měření a nedostatečnou odolností algoritmu výpočtu BPM na rušení.
\end{itemize}

Z těchto možností se zdají nejvíce pravděpodobná poslední 2 možnosti, tedy nesprávný výstup senzoru, problémy algoritmu proti rušení a časování měření.

\end{document}